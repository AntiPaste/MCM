\documentclass[12pt]{article}
\usepackage[latin1]{inputenc}
\usepackage{enumerate}
\usepackage{geometry}                % See geometry.pdf to learn the layout options. There are lots.
\geometry{letterpaper}                   % ... or a4paper or a5paper or ... 
%\geometry{landscape}                % Activate for for rotated page geometry
%\usepackage[parfill]{parskip}    % Activate to begin paragraphs with an empty line rather than an indent
\usepackage{graphicx}
\usepackage{amssymb}
\usepackage{amsmath}
\usepackage{epstopdf}
\usepackage{comment}
\usepackage{alltt}
\usepackage{url}
\usepackage{xcolor}
\DeclareGraphicsRule{.tif}{png}{.png}{`convert #1 `dirname #1`/`basename #1 .tif`.png}
\newcommand{\NB}[0]{\textcolor{red}{\textbf{!l�hde!}}}
\newcommand{\TD}[0]{\textcolor{red}{\textbf{!TODO!}}}
\newcommand{\E}[0]{\textbf{Ebola }}

\title{EBORA - Evolutively Balanced Operational Rescue Algorithm}
\author{Team 41983}
\date{}                                           % Activate to display a given date or no date

\begin{document}

\setlength{\parindent}{0cm}


\section{Introduction}

Viruses like ebola can have serious effect on global scale. The latest massive Ebola outbreak has taken serious toll in West Africa. Although scientifically the serious problem seems to pin down the medical properties of the virus and hence find the cure, some care is needed while dealing with the distribution of the vaccine to ensure the fast, humane and economical solution.

In this paper, we combine multiple celebrated mathematical models to fight the headline camper \E. First, we build a model to model the progress of Ebola outbreak based on population distribution. Then element of medicine production and distribution is introduced. We investigate how the global distribution behaviour affects large scale outcome of the spread: where the medicine is deployed and when.

Two main strategies are investigated: either medicine distributed mainly according to the wishes of regions, or 'higher being' distributes the medicines to reach globally optimised solution.


\end{document}