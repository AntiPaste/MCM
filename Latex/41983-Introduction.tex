\documentclass[12pt]{article}
\usepackage[latin1]{inputenc}
\usepackage{enumerate}
\usepackage{geometry}                % See geometry.pdf to learn the layout options. There are lots.
\geometry{letterpaper}                   % ... or a4paper or a5paper or ... 
%\geometry{landscape}                % Activate for for rotated page geometry
%\usepackage[parfill]{parskip}    % Activate to begin paragraphs with an empty line rather than an indent
\usepackage{graphicx}
\usepackage{amssymb}
\usepackage{amsmath}
\usepackage{epstopdf}
\usepackage{comment}
\usepackage{alltt}
\usepackage{url}
\usepackage{xcolor}
\DeclareGraphicsRule{.tif}{png}{.png}{`convert #1 `dirname #1`/`basename #1 .tif`.png}
\newcommand{\NB}[0]{\textcolor{red}{\textbf{!l�hde!}}}
\newcommand{\TD}[0]{\textcolor{red}{\textbf{!TODO!}}}
\newcommand{\E}[0]{\textbf{Ebola }}

\title{EBORA - Evolutively Balanced Operational Rescue Algorithm}
\author{Team 41983}
\date{}                                           % Activate to display a given date or no date

\begin{document}

\setlength{\parindent}{0cm}


\section{Introduction}

Viruses like ebola can have serious effect on global scale. The latest massive Ebola outbreak has taken serious toll in West Africa. Although scientifically the serious problem seems to pin down the medical properties of the virus and hence find the cure, some care is needed while dealing with the distribution of the vaccine to ensure a fast, humane and economical solution. 

In this paper we tackle the latter problem by exploring different strategies of distribution with mathematical models. We use our own extension of well established disease prediction models to best emulate the behaviour or Ebola. First, we build a model to model the progress of Ebola outbreak based on population distribution. Then the element of medicine production and distribution is introduced. We investigate how the global distribution behaviour affects the large scale outcome of the spread: how can Ebola be combated by efficient medication distribution methods to save lives.

To achiev our goal, first of all we needed a realistic model of how Ebola behaves and spreads in a complex network. To simplify our task we make a large scale model, where the smallest unit of measure is a single city. Since the number of cities in the world is too large to model effectively we have limited the model to about 3000 major cities, that is not too heavy to model but descriptive of the global scenario.




\end{document}
