\documentclass[12pt]{article}
\usepackage[latin1]{inputenc}
\usepackage{enumerate}
\usepackage{geometry}                % See geometry.pdf to learn the layout options. There are lots.
\geometry{letterpaper}                   % ... or a4paper or a5paper or ... 
%\geometry{landscape}                % Activate for for rotated page geometry
%\usepackage[parfill]{parskip}    % Activate to begin paragraphs with an empty line rather than an indent
\usepackage{graphicx}
\usepackage{amssymb}
\usepackage{amsmath}
\usepackage{epstopdf}
\usepackage{comment}
\usepackage{alltt}
\usepackage{url}
\usepackage{xcolor}
\DeclareGraphicsRule{.tif}{png}{.png}{`convert #1 `dirname #1`/`basename #1 .tif`.png}
\newcommand{\NB}[0]{\textcolor{red}{\textbf{!l�hde!}}}
\newcommand{\TD}[0]{\textcolor{red}{\textbf{!TODO!}}}
\newcommand{\E}[0]{\textbf{Ebola }}

\title{EBORA - Evolutively Balanced Operational Rescue Algorithm}
\author{Team 41983}
\date{}                                           % Activate to display a given date or no date

\begin{document}

\setlength{\parindent}{0cm}


\section{Strength And Weaknesses}

Since the development of an epidemic can be quite complicated and random, some harsh simplifications must be done in order to keep the computational complexity in reasonable scope. We extended a classical mathematical model with statistical flower trying to combine their strength, since they can both be a bit weak on themselves. Classical models have been in great use, so they can be deemed fairly reliable, but playing with them on a global scale can still cause substantial inaccuracies. Also fairly natural things i.e. political conflict etc. mentioned in the section Assumption could in real life a serious effect on the birth of the epidemic.





\end{document}