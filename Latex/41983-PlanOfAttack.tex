\documentclass[12pt]{article}
\usepackage[latin1]{inputenc}
\usepackage{enumerate}
\usepackage{geometry}                % See geometry.pdf to learn the layout options. There are lots.
\geometry{letterpaper}                   % ... or a4paper or a5paper or ... 
%\geometry{landscape}                % Activate for for rotated page geometry
\usepackage[parfill]{parskip}    % Activate to begin paragraphs with an empty line rather than an indent
\usepackage{graphicx}
\usepackage{amssymb}
\usepackage{amsmath}
\usepackage{epstopdf}
\usepackage{comment}
\usepackage{alltt}
\usepackage{url}
\usepackage{xcolor}
\DeclareGraphicsRule{.tif}{png}{.png}{`convert #1 `dirname #1`/`basename #1 .tif`.png}
\newcommand{\NB}[0]{\textcolor{red}{\textbf{!l�hde!}}}
\newcommand{\TD}[0]{\textcolor{red}{\textbf{!TODO!}}}
\newcommand{\E}[0]{\textbf{Ebola }}

\title{EBORA - Evolutively Balanced Operational Rescue Algorithm}
\author{Team 41983}
\date{}                                           % Activate to display a given date or no date

\begin{document}

\setlength{\parindent}{0cm}


\section{Plan of Attack}

In order to model the development of the \E outbreak and it's eradication, we combine multiple celebrated mathematical models. We recognise activity on multiple levels: city, region, and global. On the city level, we use an extension of classical SIR model \NB , SEIAR for the in-city dynamics. Region consist of cities, which interact (according to model \NB) moving the disease. Regions also interact with each other on a higher level through the most critical cities in a region. The medicine system is then added to this dynamics.

The 'World Map' is created with data about the current (around 3000) most populated cities \NB. Traffic information is used to estimate movement between cities \NB. This data in mind, a network between cities and regions is created: the \E only disperses through these routes.

Medicine factories are distributed in the biggest regions. In every region there is a vaccine supplier, which takes care of region. Supplier sends requests to the medicine factories, which try to satisfy their needs every day. Upon receiving their medicine packages, suppliers deliver the medicine to the cities. When the medicine has been shipped to the city, it's distributed from that point on at a rate depending on the size of the city.

We estimate the effectiveness of the medication strategy with a simple idea. \TD



%Different vaccination strategies are tested evolutively \TD and their expense estimated using results from similar disease spreads and vaccination costs (see \NB) \TD.






\end{document}